\usepackage{tocloft}
\usepackage{titlesec}
\usepackage{hyperref}
\usepackage{xcolor}
\usepackage{graphicx}
\usepackage{geometry}
\usepackage{listings}
\usepackage{nameref}
\usepackage{fancyhdr} % 引入页眉页脚控制包

% 页面设置
\geometry{a4paper, left=2.5cm, right=2.5cm, top=2.5cm, bottom=2.5cm}

% ===== 页码设置 =====
% 设置所有页面底部居中页码
\pagestyle{fancy} % 使用fancy页眉页脚样式
\fancyhf{} % 清除所有页眉页脚设置
\fancyfoot[C]{\thepage} % 设置页脚居中页码

% 处理章节首页的特殊样式
\fancypagestyle{plain}{
    \fancyhf{} % 清除plain样式设置
    \fancyfoot[C]{\thepage} % 设置章节首页也居中显示页码
}

% 确保其他页面样式也使用相同设置
\renewcommand{\headrulewidth}{0pt} % 去掉页眉横线
\setlength{\footskip}{15mm} % 调整页脚与正文的距离

% 标题编号深度
\setcounter{secnumdepth}{3} % 3表示包含subsubsection层级

% 标题格式
\titleformat{\section}[hang]{\bfseries\large}{\thesection}{1em}{}
\titleformat{\subsection}[hang]{\bfseries}{\thesubsection}{1em}{}
\titleformat{\subsubsection}[hang]{\bfseries}{\thesubsubsection}{1em}{} % 添加subsubsection格式

% 超链接设置
\definecolor{myblue}{RGB}{30,144,255}
\hypersetup{
    colorlinks=true,
    linkcolor=myblue,
    urlcolor=myblue,
    citecolor=myblue
}

% 代码高亮设置
\lstset{
    basicstyle=\ttfamily\small,
    backgroundcolor=\color{white},
    keywordstyle=\color{blue}\bfseries,
    commentstyle=\color{gray}\itshape,
    stringstyle=\color{red!70!black},
    numberstyle=\tiny\color{gray},
    frame=single,
    breaklines=true,
    columns=fullflexible,
    tabsize=4,
    showstringspaces=false,
    morekeywords={uint8_t, uint16_t, uint32_t, size_t},
}

% 自定义颜色
\definecolor{githubgray}{RGB}{245,245,245}
\definecolor{githubblue}{RGB}{36,41,46}
\definecolor{githubgreen}{RGB}{34,139,34}
\definecolor{githubred}{RGB}{255,85,85}

% 标题、作者、日期
\title{Learn PBL with C}
\author{Musubi}
\date{\today}
