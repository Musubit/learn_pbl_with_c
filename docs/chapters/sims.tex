\chapter{学生信息管理系统}

\section{项目概述}

一个基于C语言的简单学生信息管理系统,使用单向链表实现数据存储。

\section{项目设计}

\subsection{数据结构设计}\label{sub:data_structure_design}

系统使用单链表来存储学生信息,每个节点包含以下字段:
\begin{itemize}
    \item $id$: 学生ID
    \item $name$: 姓名
    \item $age$: 年龄
    \item $score$: 成绩
    \item $next$: 指向下一个节点的指针
\end{itemize}

\subsection{模块设计}

系统主要分为以下几个模块: 

\begin{enumerate}
    \item \textbf{学生节点管理模块}: 负责学生信息的增删改查操作
    \item \textbf{课程链表操作模块}: 负责课程信息的增删改查操作
    \item \textbf{用户操作模块}: 负责接受用户输入调用相应的功能并显示结果
    \item \textbf{主控制模块}: 负责系统的整体控制和调度
\end{enumerate}

\subsection{功能设计}

系统提供以下功能:
\begin{itemize}
    \item \textbf{添加学生信息}: 通过输入学生的基本信息(如ID、姓名、年龄、成绩)来添加新学生。
    \item \textbf{删除学生信息}: 通过输入学生ID来删除对应的学生信息。
    \item \textbf{修改学生信息}: 通过输入学生ID来修改对应的学生信息。
    \item \textbf{查询学生信息}: 通过输入学生ID来查询对应的学生信息。
    \item \textbf{显示所有学生信息}: 显示系统中所有学生的基本信息。
\end{itemize}

\section{详细设计}

\subsection{数据结构实现}

实现上文的\ref{sub:data_structure_design}: \nameref{sub:data_structure_design}中定义的学生信息节点结构体,使用C语言的结构体和指针来实现单向链表。

\begin{lstlisting}[language=C, caption=学生信息节点结构体]
typedef struct Student {
    int id;
    char name[50];
    int age;
    float score;
    struct Student* next;
} Student;
\end{lstlisting}

\subsection{核心函数实现}

\subsubsection{创建学生节点}
\begin{lstlisting}[language=C, caption=创建学生节点函数]
Student* createStudent(int id, const char* name, int age, float score) {
    Student* newStudent = (Student*)malloc(sizeof(Student));
    if (newStudent == NULL) {
        printf("内存分配失败\n");
        return NULL;
    }
    newStudent->id = id;
    strcpy(newStudent->name, name);
    newStudent->age = age;
    newStudent->score = score;
    newStudent->next = NULL;
    return newStudent;
}
\end{lstlisting}

\begin{itemize}
    \item \textbf{\textcolor{mydeepgreen}{功能}}: 创建一个新的学生节点并返回指向该节点的指针。
    \item \textbf{\textcolor{mydeepgreen}{参数}}: 学生ID、姓名、年龄、成绩。
    \item \textbf{\textcolor{mydeepgreen}{返回值}}: 返回新创建的学生节点指针。
\end{itemize}
